% Copyright 2007 by Till Tantau
%
% This file may be distributed and/or modified
%
% 1. under the LaTeX Project Public License and/or
% 2. under the GNU Public License.
%
% See the file doc/licenses/LICENSE for more details.



\documentclass{beamer}

%
% DO NOT USE THIS FILE AS A TEMPLATE FOR YOUR OWN TALKS�!!
%
% Use a file in the directory solutions instead.
% They are much better suited.
%


% Setup appearance:

\usetheme{Darmstadt}
%\usetheme{CambridgeUS}
%\usetheme{Berkeley}
%\usetheme{Hannover}
%\usefonttheme[onlylarge]{structurebold}
\usefonttheme{professionalfonts}
%\usefonttheme{default}
\setbeamerfont*{frametitle}{size=\normalsize,series=\bfseries}
\setbeamertemplate{navigation symbols}{}
\setbeamercovered{transparent=30}
\setbeamertemplate{footline}{
    
    \leavevmode%
    \hbox{%
    \begin{beamercolorbox}[wd=.4\paperwidth,ht=2.25ex,dp=1ex,center]{author in head/foot}%
        \usebeamerfont{author in head/foot}\insertshortauthor
    \end{beamercolorbox}%

    \begin{beamercolorbox}[wd=.6\paperwidth,ht=2.25ex,dp=1ex,center]{title in head/foot}%
        \usebeamerfont{title in head/foot}\insertshorttitle\hspace*{3em}
        \insertframenumber{} / \inserttotalframenumber\hspace*{1ex}
    \end{beamercolorbox}}%
    \vskip0pt% 
    
    
}

% Standard packages

\usepackage[english]{babel}
\usepackage[latin1]{inputenc}
\usepackage{times}
\usepackage[T1]{fontenc}
\usepackage{libertine}
\usepackage{graphicx}
%\usepackage{natbib}
\usepackage{enumerate}
\usepackage{biblatex}


\usepackage{epstopdf}
\usepackage{tabularx}
\usepackage{subfigure}



% Setup TikZ

\usepackage{tikz}
\usetikzlibrary{arrows}
\tikzstyle{block}=[draw opacity=0.7,line width=1.4cm]
\addbibresource{target.bib}
\let\oldfootnotesize\footnotesize
\renewcommand*{\footnotesize}{\oldfootnotesize\tiny}
\graphicspath{{../img/svg/}}
%change the font here
\sc

% Author, Title, etc.

\title[ 2015 Group Meeting Presentation ] 
{%
Detecting Hardware Trojans Using Backside Optical Imaging of Embedded Watermarks
  %
}

\author[Boyou Zhou]
{
  \textit{Boyou Zhou\inst{*}}
}

\institute[Boston University, MA]
{
  \inst{*}
  Boston Univeristy, MA
}

\date[Spring 2015]
{Created on Feb 17 2015, Modified on \today}

% The main document

\begin{document}

\begin{frame}
  \titlepage
\end{frame}

\begin{frame}{Outline}
  \tableofcontents
\end{frame}

\AtBeginSubsection{
    \frame<beamer>{ 
    \frametitle{Outline}   
    \tableofcontents[currentsection,currentsubsection]
    }
}

\section{Introduction}
\subsection{Hardware Security}
\begin{frame}{Hardware Security}
    \begin{itemize}
    \item [*] A generic IC chip development cycle includes specification, design,
    fabrication, testing and packaging phases. 
\pause
    \item [*] During these various phases, the IC chips face threats from
    hardware Trojans, IP privacy and IC overbuilding, reverse engineering,
    side-channel analysis and IC counterfeiting.\footfullcite{rostami2014primer}
    \end{itemize}
\end{frame}

\begin{frame}{cost for hardware security}
    \begin{itemize}
    \item [*] Not only hardware security problems may affect everyday consumer
    electronics, but also in critical defense and core support
    systems.
\pause
    \item [*] It was estimated that the cost of hardware security for G20 nations
    was \$450 to \$650 billion in 2008 and will grow to \$1.2 to 1.7 trillion in
    2015.\footfullcite{guin2013counterfeit}
    \footfullcite{economics2011estimating}
    \footfullcite{tehranipoortrusthub}
    \end{itemize}
\end{frame}

\subsection{Hardware Trojans}
\begin{frame}{Hardware Trojans}
    \begin{itemize}
    \item Hardware Trojans are modifications to original circuitry inserted by
    adversaries, in order to exploit hardware to gain access to data or
    software running on the chips.
    \footfullcite{tehranipoor2010survey}
\pause
    \item Wang, Tehranipoor, and Plusquellic developed the first detailed for
    hardware Trojans in 2008.
    \footfullcite{wang2008detecting}
    \end{itemize}
\end{frame}

\section{Related Work}
\subsection{Anti-Trojan Techniques}
\begin{frame}{Related Work}
    Electrical Methods and Side Channel Detections are the basic methods for
    anti-trojan techniques.
    \begin{itemize}
        \item \textit{Electrical Methods}
        Delay-based Analysis, 
        \footfullcite{li2008speed}
        \footfullcite{jin2008hardware}
        Power Analysis.
        %\footfullcite{rad2008power}
        %\footfullcite{alkabani2009consistency}
        %\footfullcite{potkonjak2009hardware}
        \footfullcite{wei2012scalable}
        \item \textit{SCD Methods}
        Thermal Analysis, 
        \footfullcite{forte2013temperature}
        Sound Analysis,
        Electromagnetic Analysis.
        \footfullcite{hu2013high}
    \end{itemize}
\end{frame}

\subsection{Post Silicon Trojan Detections}
    \begin{frame}{PUF}
        One of the most popular techniques in hardware security is Physically
        Unclonable Functions. Listed below is how PUFS are built.
        \begin{itemize}
            \item [*] Engineer two identical electrical paths. They can be logic
            units, memory units or clock trees.
            \item [*] With the information of the manufacture, design CRPs
            (challenge and response pairs). The challenges are the logic inputs
            and responses are the delay differences between two paths.  inputs
            that will have unique delay responses.
            \item [*] Chips with correct response can be considered to be not
            tampered.
            \footfullcite{suh2007physical}
        \end{itemize}
    \end{frame}

    \begin{frame}{Our Technique}
        \begin{itemize}
            \item Engineered the layout of the fill cells in a standard
            cell library. Any replacement, modification or rearrangement of these
            fill cells to add HTs can be easily detected by backside imaging.
            \item Designed hardware blocks using standard Cadence toolflows in
            45 nm technology, proved with backside imaging, and compared it with
            power analysis.
        \end{itemize}
    \end{frame}

\section{Optical Watermarks}
\subsection{Filler Cell Design}
\begin{frame}{Design of Filler Cell}
    The new filler cell is filled with metal without connecting to the top and
    bottom power rail.
\end{frame}

\subsection{Customed Filler Cell Comparison}
\begin{frame}{Optical Response}
    \begin{figure}
    \includegraphics[width=0.5\columnwidth]{uni_response.pdf}
        \caption{\small{ The reflectance spectrum of functional gates and
        fill cells, computed via FDTD simulations. The response is computed for
        both X and Y polarizations of the illuminating field (solid and dashed
        lines, respectively). }}
    \label{fig:single_cell_response}
    \end{figure}
\end{frame}

\section{Hardware Trojan Detection}
\subsection{Backside Imaging}
\begin{frame}{Backside Imaging}
    \begin{figure}[t]
        \subfigure[]{
            \includegraphics[trim=0mm 0mm 0mm 0mm,clip,width=1.3in]{layout.png}
            \label{layout}
        }~
        \subfigure[]{
            \includegraphics[trim=8mm 8mm 8mm 8mm,clip,width=1.8in]{mod/Free.pdf}
            \label{multiple_gates_without_Trojans}
        }~
        \caption{\small{
            (a) Physical layout of a $10\mu m \times 10\mu m$ region of the AEST100
            hardware block. (b) Backside image (reflectance value) of the $10\mu m
            \times 10\mu m$ region. The fill cells have the highest reflectance. 
        }}
        \label{fig:backside-img1}
    \end{figure}
\end{frame}

\begin{frame}{Trojan Tampering}
\begin{figure}[t]
    \centering
    \subfigure[]{
        \includegraphics[trim=8mm 8mm 8mm 8mm,clip,width=1.4in]{mod/In.pdf}
        \label{multiple_gates_with_changes_to_filler_cells}
    }~
    \subfigure[]{
        \includegraphics[trim=8mm 8mm 8mm 8mm,clip,width=1.4in]{shi/In.pdf}
        \label{multiple_gates_with_shifts_to_filler_cells}
    }~
    \subfigure[]{
        \includegraphics[trim=8mm 8mm 8mm 8mm,clip,width=1.4in]{rep/In.pdf}
        \label{multiple_gates_with_changes_to_none_filler_cells}
    }~
    \caption{\small{
        (a) Backside image of the $10\mu m \times 10\mu m$
        region when the fill cells are replaced with functional gates
        (b) Backside image of the $10\mu m \times 10\mu m$
        region when the bottom 3 rows are shifted by 5 $\mu$m to the left make room for
        cells. 
        (c) Backside image of the $10\mu m \times 10\mu m$
        region when the functional cells are replaced different set of
        functional cells.}}
    \label{fig:backside-img2}
\end{figure} 
\end{frame}

\subsection{Improving Dectection Rate}
\begin{frame}{Image Correlation}
    Pixel by pixel correlation formula is listed below.
    \begin{equation}
    %C(k,l) = \sum\limits_{i=0}^{q-1}{\sum\limits_{j=0}^{r-1}{M(i,j)N(i-k,j-l)}}
    r = \frac{\sum\limits_{i}\sum\limits_{j}(M_{ij}-\bar{M})(N_{ij}-\bar{N})}
        {\sqrt{(\sum\limits_{i}\sum\limits_{j}(M_{ij}-\bar{M})^2)
        (\sum\limits_{i}\sum\limits_{j}(N_{ij}-\bar{N})^2)}}
    \label{eq:Defination}
    \end{equation}
\end{frame}

\begin{frame}{Threshold Change}
\begin{figure}
    \includegraphics[width=0.5\columnwidth]{thresholdDecision.pdf}
    \label{threshold_decision}
    \caption{Detection error rate versus Threshold values for a fixed SNR of $10$.}
\end{figure}
\end{frame}

\begin{frame}{AEST100 Simulation Test}
\begin{figure}[t]
    \centering
    \subfigure[]{
        \includegraphics[width=1.5in]{real/Free.PNG}
        %\includegraphics[width=\columnwidth]{real/Free.pdf}
        \label{50by50_without_trojans}
    }
    \subfigure[]{
        \includegraphics[width=1.5in]{real/In.PNG}
        \label{50by50_with_trojans}
    }
    \caption{
        (a) Backside image (reflectance values) of a Trojan-free $50\mu m \times 50\mu
        m$ region of the AEST100 hardware block.
        (b) Backside image (reflectance values) of the same $50\mu m \times 50\mu
        m$ region of the AEST100 hardware block with {\color{black}CDMA private
        key disclosure} type of HT inserted in it.}
    \label{imaging_result}
\end{figure}
\end{frame}

\begin{frame}{Signal to Noise Ratio}
\begin{figure}
	\centering
    \includegraphics[width=1.0\textwidth]{snr.pdf}
    \label{fig:snr}
    \caption{
        {\color{black}Testbench AES and PIC Trojan Detection Rate under different
    Signal-to-Noise Ratio
    (a) is AEST100
    (b) is AEST200
    (c) is AEST1000
    (d) is PICT100
    (e) is PICT200
    (f) is PICT300}
    }
\end{figure}
\end{frame}

\section{Evaluation}
\subsection{Process Variation}
\begin{frame}{Process Variation Impacts}
\begin{figure}[!t]
    \begin{center}
        \includegraphics[width=1\textwidth]{fillercells.pdf}
        \caption{Impact of process variations on the reflectance signal for various wavelengths.}
        \label{fig:PV}
    \end{center}
\end{figure}
\end{frame}

\subsection{Power and Area Analysis}
\begin{frame}{Power Comparison}
    \resizebox{\textwidth}{!}{
    \begin{tabular}{c c c c c c c}
        Testbench & Leakage Power & Total Power & Leakage Power  & Total Power  & Leakage        & Total  \\
                  &               &             &(with Trojan)   & (with Trojan)& Percentage (\%)& Percentage (\%) \\
	\hline
        AES100    &   2.813       & 172.2       & 2.859          & 174          & 1.64           & 1.05 \\
        AES200    &   2.862       & 167.8       & 2.885          & 169.5        & 0.8            & 1.0 \\
        AES1000   &   2.813       & 172.2       & 2.859          & 174          & 1.64           & 1.05 \\
        PIC100    & 0.03811       & 0.5248      & 0.04196 & 0.6803 & 1.0 & 29 \\
        PIC200    & 0.03811       & 0.5248 & 0.03857 & 0.704 & 1.21 & 34\\
        PIC300    & 0.03617       & 0.2759 & 0.04196 & 0.4416 & 16 & 60 \\
    \end{tabular}
    \label{tab:power}
    }
    \begin{table}
    \caption{Power consumption (in $mW$) in various hardware blocks. The baseline designs
    of AES100, AES200 and AES1000 is the same, the inserted HTs are different.
    Similarly, the baseline design of PIC100, PIC200 and PIC300 are the same but
    the inserted HTs are different.}
    \end{table}
\end{frame}

\begin{frame}{Area Comparison}
\begin{table}[t]
    %\resizebox{\textwidth}{!}{%
    \begin{center}
    \begin{tabular}{c c c c}
        Testbench & Area without & Trojan     & Trojan Area      \\
                  & Trojans      & Area       & Percentage (\%)  \\
	\hline
        AES100    & 274177.6     & 253.2      & 0.0923 \\
        AES200    & 274177.6     & 169.5      & 0.0618 \\
        AES1000   & 274177.6     & 251.1      & 0.0915 \\
		PIC100 	  & 4215.0 & 351 & 8.33 \\
		PIC200    & 4215.0 & 89.6 & 2.13 \\
		PIC300    & 4215.0 & 253.2 & 6.01 \\
    \end{tabular}
    \caption{Area (in $\mu m^2$) occupied by various hardware blocks.}
    \label{tab:area}
    %}
    \end{center}
\end{table}
\end{frame}

\end{document}
